% abstract.tex - resumen de la tesis de grado

\begin{abstract}


La M'aquina Abstracta \emph{MAPiCO} es la implementaci'on de una m'aquina abstracta para el c'alculo computacional \emph{PiCO} \footnote{\emph{PiCO}: C'alculo de objetos y restricciones}\cite{pico-ant}, esta m'aquina sigue las especificaciones y las \mbox{reglas} de reducci'on de dicho c'alculo, que es la base del modelamiento matem'atico, \mbox{s'olido} y formal del lenguaje de programaci'on \emph{Cordial} \footnote{\emph{CORDIAL}: Constraints Objects Relations Defining an Iconic Applicative Languaje} \cite{visual-model-cordial}.\\
\\
La actual implementaci'on de \emph{MAPiCO} \cite{buss.heredia:mapico} es correcta, formal y cumple con los \mbox{requisitos} y prop'ositos para los cuales fue creada, sin embargo, el lenguaje de programaci'on en la que fue implementada, \emph{JAVA}, hace que el tiempo de ejecuci'on se vea afectado ya que por ser un lenguaje interpretado y no compilado compromete la eficiencia \cite{maq-abs}; adicionalmente esta implementaci'on no soporta las modificaciones del C'alculo \emph{PiCO} \cite{Integ-Const-Conc-Obj}.\\
\\
Este trabajo muestra una reimplementaci'on de la M�quina Abstracta \emph{MAPiCO}, considerando las modificaciones realizadas al c�lculo \cite{Integ-Const-Conc-Obj} y la extensi�n de \mbox{instrucciones} de la m�quina a trav�s de m�dulos din�micos, que garantiza en un futuro no redise~nar la M'aquina Abstracta. As� mismo se ofrece una alternativa de \mbox{eliminaci'on} de variables locales de procesos que ya se han ejecutado por completo y que pueden no necesitarse m�s en el transcurso de la ejecuci�n.
\end{abstract}
