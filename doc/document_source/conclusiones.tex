\chapter{CONCLUSIONES}
\label{capconclusiones}
En el presente trabajo se reimplement'o la m'aquina abstracta \emph{MAPiCO}, m'aquina soportada por el c'alculo l'ogico, correcto y formal \emph{PiCO}. La reimplementaci'on de \emph{MAPiCO} surge de la necesidad de mejorar la eficiencia, contemplar las modificaciones del c'alculo y extender de manera din'amica el conjunto de instrucciones de la m'aquina.\\\\
Aunque la implementaci'on inicial cumple con los prop'ositos y requisitos para los cuales fue creada, el lenguaje en el que fue implementada, \emph{JAVA}, hace que se pierda eficiencia en el tiempo de ejecuci'on de la m'aquina, ya que por ser un lenguaje interpretado es mas lento que un lenguaje compilado, lo cual se evidencia en las pruebas realizadas.\\\\
La actual implementaci'on se realiz'o en el lenguaje de programaci'on \emph{C}, considera directrices de precompilaci'on  y modularidad en el desarrollo, lo que garantiza tiempos de ejecuci'on mucho mas eficientes, funcionalidad modular y practicidad en la modificaciones que deban realizarse en un futuro a la m'aquina.\\\\
Debido a que la reimplementaci�n de \emph{MAPiCO} soporta la extensibilidad en cuanto al conjunto de instrucciones, se hace posible implementar cualquier modificaci�n al c�lculo \emph{PiCO}, de forma modular y sin reimplementar de nuevo la M�quina Abstracta.\\\\
Otra de las caracter�sticas de la actual implementaci'on es que gracias a su extensibilidad, flexibilidad y modularidad puede ejecutarse como una m'aquina de m'aquinas. En este caso, se podr�a implementar la funcionalidad de otra m�quina mediante los \emph{PlugIn's} y ejecutarlos en \emph{MAPiCO}; la M�quina Abstracta \emph{LMAN} \cite{hurtado.munoz:lman} podr�a implementarse bajo este criterio.\\\\
La alternativa de Eliminaci�n de Variables planteada en el capitulo {\bf \ref{capmapico}} (secci�n \ref{secElimVar}) podr�a significar una mejora considerable del \emph{Store}, con respecto a la propagaci�n de variables.\\\\
Se realizaron pruebas que determinaron la eficiencia de la reimplementaci'on de la m'aquina abstracta \emph{MAPiCO}:
\begin{itemize}
\item Pruebas en las que se verific'o las complejidades de cada una de las funciones utilizadas para las estructuras de listas, 'arboles, algoritmos de b'usqueda, de almacenamiento de memoria, de precompliaci'on y modularidad. Cada una de las complejidades de estas funciones se encuentran en en el {\bf capitulo \ref{capimpl}} y en los archivos de cabecera de la implementaci'on.
\item Pruebas de carga de archivos de programa de gran tama�o en la memoria est�tica y de \emph{PlugIn's} en tiempo de ejecuci�n; que confirmaron la capacidad de almacenamiento y eficiencia en modularidad/extensibilidad de la reimplementaci�n de \emph{MAPiCO} (capitulo \ref{capimpl}).
\item Pruebas funcionales que determinaron la correcta reimplementaci�n de la M�quina Abstracta \emph{MAPiCO}.
\item Las pruebas realizadas en el {\bf capitulo \ref{capprueba}} demostraron que:
\begin{itemize}
\item El \textbf{Escenario 2}, determinado por \emph{MAPICO JAVA} y \emph{Store C} es un $66 \%$ m�s eficiente que el \textbf{Escenario 1} constituido por \emph{MAPICO JAVA} y \emph{Store JAVA}.
\item El \textbf{Escenario 3}, determinado por \emph{MAPICO C} y \emph{Store C} es un $99 \%$ m�s eficiente que el \textbf{Escenario 1} constituido por \emph{MAPICO JAVA} y \emph{Store JAVA}.
\item El \textbf{Escenario 3}, determinado por \emph{MAPICO C} y \emph{Store C} es un $99 \%$ m�s eficiente que el \textbf{Escenario 2} constituido por \emph{MAPICO JAVA} y \emph{Store C}.
\end{itemize}
Con respecto a las anteriores pruebas se puede concluir que \emph{MAPiCO C} cumple con los prop�sitos para los cuales fue reimplementada: es eficiente ya que presenta los m�nimos tiempos de ejecuci�n, es extensible y modular en cuanto al conjunto de instrucciones y brinda una alternativa de eliminaci�n de variables que no se necesitan mas en la ejecuci�n.
\end{itemize}
\nocite*